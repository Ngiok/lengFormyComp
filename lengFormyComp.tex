\documentclass{article}
\usepackage{amsmath}

\title{Lenguajes Formales y Computabilidad\linebreak Definiciones, lemas y teoremas}
\author{Valentino Beorda}
\date{April 2022}

\begin{document}

\maketitle

\section{Funciones $\Sigma$-mixtas}

\begin{flushleft}

Una función es un conjunto $f$ de paers ordenados con la siguiente propiedad:
\begin{itemize}
\item Si $(x, y) \in f$ y $(x, z) \in f$, entonces  $y=z$.
\end{itemize}

$D_f$: dominio de $f$ = $\{x : (x, y) \in f$ para algún $y\}$\linebreak
$I_f$: imagen de $f$ = $\{y : (x, y) \in f$ para algún $x\}$\linebreak

\textbf{Igualdad de funciones:} Sean $f$ y $g$ funciones. Entonces $f = g$ sii $D_f = D_g$ y para cada $x \in D_f$ se tiene que $f(x) = g(x)$.\linebreak

\textbf{Definición:}
Sea $\Sigma$ un alfabeto finito. Dada una función $f$, diremos que $f$ es $\Sigma$-mixta si cumple las siguientes propiedades:
\begin{enumerate}
\item Existen $n$, $m$ $\geq$ 0, tales que $D_f$ $\subseteq \omega^n \times \Sigma^{*m}$
\item $I_f \subseteq \omega$ ó $I_f \subseteq \Sigma^*$.
\end{enumerate}

\textbf{Lemma:} Supongamos que $\Sigma \subseteq \Gamma$ son alfabetos finitos. Entonces si $f$ es una función $\Sigma$-mixta, $f$ es $Gamma$-mixta\linebreak

\textbf {Funciones Útiles:}\linebreak

Sucesor: $Suc$: $\omega \rightarrow \omega$\linebreak
\hspace*{14ex} $n \mapsto n+1$\linebreak

Predecesor: $Pred: $ \textbf{N} $\rightarrow \omega$\linebreak
\hspace*{20ex} $n \mapsto n-1$\linebreak

Derecha sub a: $d_a: \Sigma^* \rightarrow \Sigma^*$\linebreak
\hspace*{21ex} $\alpha \mapsto \alpha a$\linebreak

Proyecciones: \linebreak
para un $i$ tal que $1 \geq i \geq n$,\linebreak
$p_i^{n, m}: \omega^n \times \sigma^{*m} \rightarrow \omega$\linebreak
\hspace*{8ex} $(\overrightarrow{x}, \overrightarrow{\alpha}) \mapsto x_i$\linebreak
Para un $i$ tal que $n \geq i \geq n+m$, \linebreak
$p_i^{n, m}: \omega^n \times \sigma^{*m} \rightarrow \omega$\linebreak
\hspace*{8ex} $(\overrightarrow{x}, \overrightarrow{\alpha}) \mapsto \alpha_{i-n}$\linebreak

Constantes: $C_k^{n, m}: \omega^n \times \Sigma^{*m} \rightarrow \omega$ \hspace*{13ex}$C_\alpha^{n, m}: \omega^n \times \Sigma^{*m} \rightarrow
\Sigma^*$\linebreak
\hspace*{22ex} $(\overrightarrow{x}, \overrightarrow{\alpha}) \mapsto k $ \hspace*{22ex} $(\overrightarrow{x}, \overrightarrow{\alpha}) \mapsto \alpha $\linebreak\linebreak

\textbf{Tipo de una función mixta:} Dada $f$ una función $\Sigma$-mixta, si $n, m \in \omega$ son tales que $D_f \subseteq \omega^n \times \Sigma^{*m}$ y además $I_f \subseteq \omega$, diremos $f$ es una función de tipo $(n, m, \#)$; si $I_f \subseteq \Sigma^*$, entonces $f$ es de tipo $(n, m, *)$. \linebreak

\textbf{Predicados $\Sigma$-mixtos:} Un predicado $\Sigma$ mixto es una función $f$ la cual es $\Sigma$-mixta y además cumple $I_f \subseteq \{0,1\}$.\linebreak

\textbf{Función identidad:} Dado un conjunto A, a la función:
\begin{center}
$A \rightarrow A$\linebreak
$a \mapsto a$
\end{center}
La denotaremos con $Id_A$ y la llamaremos la función identidad sobre A. Notar que $Id_a = \{(a, a): a \in A\}$ \linebreak.

\textbf{Composición de funciones:} Dadas funciones $f$ y $g$ definamos la función $f \circ g$ de la siguiente manera:
\begin{itemize}
\item $D_{f \circ g} = \{ e \in D_g : g(e) \in D_f\}$
\item $f \circ g(e) = f(g(e))$
\end{itemize}

Notar que $f \circ g = \{(u, v): \exists$ z tal que $ (u, z) \in g$ y $(z, v) \in f\}.$\linebreak

\textbf{Lemma:} $f \circ g \neq \emptyset$ si y solo si $I_g \cap D_f \neq \emptyset.$\linebreak

\textbf{Lemma:} Si $I_g \in D_f \Rightarrow D_{f \circ g} = D_g$ \linebreak

\textbf{Funciones de la forma $[f_1,...,f_n]$}: Dadas funciones $f_1, ..., f_n$ con $ n \geq 2$, definimos la función $[f_1, ..., f_n]$ de la siguiente forma:
\begin{itemize}
\item $D_{[f1, ..., fn]} = D_{f_1} \cap ... \cap D_{f_n}$
\item $[f_1, ..., f_n](e) = (f_1(e), ..., f_n(e))$
\end{itemize}
Notar que $I_{[f_1, ..., f_n]} \in I_{f_1} \times ... \times I_{f_n}.$\linebreak

\textbf{Función inyectiva:} Una función $f$ es inyectiva cuando \textbf{no} se da que $f(a) = f(b)$ $\forall$ $a, b \in D_f$ con $a \neq b$.\linebreak
\textbf{Función suryectiva:} Una función $f: A \rightarrow B$ es suryectiva cuando $I_f = B$.\linebreak
\textbf{Función biyectiva:} Una función $f$ es biyectiva, cuando $f$ es inyectiva y suryectiva.\linebreak
\textbf{Función inversa:} Si $f$ es biyectiva, entonces se puede definir $f^{-1}: B \rightarrow A$, de la siguiente manera:
\begin{center}
    $f^{-1}(b) = $ único $a \in A$ tal que $f(a) = b$
\end{center}
La función $f^{-1}$ será llamada inversa de $f$. Notar que $f \circ f^{-1} = Id_B$ y $f^{-1} \circ f = Id_A$.\linebreak

\textbf{Lemma:} Supongamos $f: A \rightarrow B$ y $g: B \rightarrow A$ son tales que $f \circ g = Id_B$ y $g \circ f = Id_A$. Entonces $f$ y $g$ son biyectivas, $f^{-1} = g$ y $g^{-1} = f$.\linebreak

\textbf{Conjuntos $\Sigma$-mixtos:} Un conjunto $S$ es llamado $\Sigma$-mixto si existen $n, m \in \omega$ tales que $S \subseteq \omega^n \times \Sigma^{*m}$.\linebreak

\textbf{Tipo de un conjunto $\Sigma$-mixto:} Dado un conjunto $\Sigma$-mixto $S$, si $n, m \in \omega$ son tales que $S \subseteq \omega^n \times \Sigma^{*m}$, entonces diremos que $S$ es de tipo $(n, m)$.\linebreak

\textbf{Notación Lambda:} \linebreak

Para que una expresión $E$ pueda ser utilizada en la notación lambda, deberá cumplir alguna de estas 2 propiedades:
\begin{enumerate}
    \item Los valores que asuma $E$ deberán ser elementos de $\omega$
    \item Los valores que asuma $E$ deberán ser elementos de $\Sigma^*$.
\end{enumerate}

\textbf{Definión de $\lambda_{x_1, ..., x_n, \alpha_1, ..., \alpha_m}[E]$:} Sea $x_1, ..., x_n$ una lista de variables todas distintas tal que las variables numéricas que ocurren en $E$ estén todas contenidas en $x_1, ..., x_n$ y que las variables alfabéticas que ocurren en $E$, estén todas contenidas en $\alpha_1, ..., \alpha_m$. Entonces $\lambda_{x_1, ..., x_n, \alpha_1, ..., \alpha_m}[E]$ denotará la función dada por:
\begin{enumerate}
    \item El dominio de $\lambda_{x_1, ..., x_n, \alpha_1, ..., \alpha_m}[E]$ es el conjunto de las $(n+m)$-uplas $(k_1, ..., k_n, \beta_1, ..., \beta_m)$ tales que $E$ está definida cuando le asignamos a cada valor $x_i$ el valor $k_i$ y a cada $\alpha_i$ el valor $\beta_i$.
    \item $\lambda_{x_1, ..., x_n, \alpha_1, ..., \alpha_m}[E](k_1, ..., k_n, \beta_1, ..., \beta_m)$ = valor que asume o representa $E$ cuando le asignamos a cada $x_i$ el valor $k_i$ y a cada $\alpha_i$ el valor $\beta_i$.
\end{enumerate}

Notar que para $S \subseteq \omega^n \times \Sigma^{*m}$ se tiene que $\chi_S^{\omega^n \times \Sigma^{*m}} = \lambda_{x_1, ..., x_n, \alpha_1, ..., \alpha_m}[(\overrightarrow{x}, \overrightarrow{\alpha}) \in S]$.\pagebreak

\section{Codificaciones}

Usaremos $\omega^N$ para denotar el conjunto de todas las infinituplas con coordenadas en $\omega$. Es decir: $\omega^N$ = $\{(s_1, s_2,...): s_i \in \omega$ $\forall i \geq 1\}$. Definamos el siguiente subconjunto de $\omega^N$:
\begin{center}
    $\omega^{[N]} = \{(s_1, s_2, ...) \in \omega^N : $ hay un $n \in N$ tal que $s_i = 0$, $\forall i \geq n\}$
\end{center}

Notar que $\omega^{[N]} \neq \omega^N$. Además, notar que $(s_1, s_2, ...) \in \omega^{[N]}$ sii una cantidad finita de coordenadas $(s_1, s_2, ...)$ son no nulas ($\{i : s_i \neq 0 \}$ es finito).

Definamos:\linebreak
\hspace*{15ex}$pr: N \rightarrow \omega$\linebreak
\hspace*{20ex}$n \mapsto n$-esimo número primo\linebreak

\textbf{Teorema:} Para cada $x \in N$, hay una única infinitupla $(s_1, s_2, ...) \in \omega^{[N]}$ tal que:
\begin{center}
    $\prod_{i=1}^{\infty} pr(i)^{s_i}$
\end{center}
Notar que en esta productoria solo una cantidad finita de factores son distintos de 1.\linebreak

\textbf{Lemma:} Si $p, p_1, ..., p_n$ son números primos (con $n \geq 1$) y $p$ divide a $p_1, ..., p_n$, entonces $p = p_i$, para algún $i$.\linebreak

Dada una infinitupla $(s_1, s_2, ...) \in \omega^{[N]}$ usaremos $\left\langle s_1, s_2, ...\right\rangle$ para denotar al número $\prod_{i=1}^{\infty} pr(i)^{s_i}$. \linebreak

Dado $x \in N$, usaremos $(x)$ para denotar a la única infinitupla $(s_1, s_2, ...) \in \omega^{[N]}$ tal que $x = \left\langle s_1, s_2, ...\right\rangle = \prod_{i=1}^{\infty} pr(i)^{s_i}$. Además, para $i \in N$, usaremos $(x)_i$ para denotar a $s_i$ de dicha infinitupla. Es decir que:
\begin{enumerate}
    \item $(x) = ((x)_1, (x)_2, ...)$
    \item $(x)_i$ es el exponente de $pr(i)$ en la factorización de $x$ como producto de primos.
    \item $\left\langle (x)_1, (x)_2, ...\right\rangle = x, \forall x \in N$.
    \item $\forall (s_1, s_2, ...) \in \omega^{[N]}$, se tiene que $(\left\langle s_1, s_2, ... \right\rangle)_i = s_i, \forall i \in N$. Es decir que $(\left\langle s_1, s_2, ... \right\rangle) = (s_1, s_2, ...)$. 
\end{enumerate}

\textbf{Teorema:} Las funciones:\linebreak
\hspace*{10ex}$N \rightarrow \omega^{[N]}$ \hspace*{30ex} $\omega^{[N]} \rightarrow N$\linebreak
\hspace*{10ex} $x \mapsto (x) = ((x)_1, (x)_2, ...)$ \hspace*{8ex} $(s_1, s_2, ...) \mapsto \left\langle s_1, s_2, ... \right\rangle$\linebreak
son biyecciones una inversa de la otra.\linebreak

\textbf{Lemma:} Dados $x, i \in N$ se tiene que
\begin{center}
    $(x)_i = max_t(pr(i)^t)$ divide a $x$
\end{center}

Definamos la función $Lt: N \rightarrow \omega$ de la siguiente manera:
\hspace*{20ex}$Lt(x) = max_i(x)_i \neq 0$ si $x \neq 1$\linebreak
\hspace*{20ex}$Lt(x) = 0$ si $x = 1$\linebreak

\textbf{Lemma:} $\forall x \in N$, $x = \prod_{i=1}^{Lt(x)} pr(i)^{(x)_i}$\linebreak

\begin{Large}\textbf{Órdenes totales}\end{Large}\linebreak

\textbf{Definición:} Una relación binaria $R$ sobre un conjunto $A$ es llamada un orden $parcial$ sobre A si cumple las siguientes 3 propiedades:
\begin{enumerate}
    \item Reflexividad: $xRx, \forall x \in A$
    \item Transitividad: $xRy$ y $yRz$ implica $xRz, \forall x, y, z \in A$
    \item Antisimetría: $xRy$ y $yRx$ implica $x = y, \forall x, y \in A$
\end{enumerate}

\textbf{Definición:} Por un orden total sobre A entenderemos un orden parcial $\leq$ sobre A el cual cumple:
\begin{itemize}
    \item $a \leq b$ o $b \leq a$, $\forall a, b \in A$ (tricotomía)\linebreak
\end{itemize}

\begin{Large}\textbf{Ordenes Naturales sobre $\Sigma^*$}\end{Large}\linebreak

El alfabeto que usaremos tendrá todos los numerales menos el 0 y además tendrá un símbolo para denotar al número diez, el símbolo d. Es decir:
\begin{center}
    $\widetilde{Num} = \{1, 2, 3, 4, 5, 6, 7, 8, 9, d\}$
\end{center}
Representaremos a los números de $\omega$ con la siguiente lista infinita de palabras de $\widetilde{Num}$:\linebreak
$\varepsilon$, 1, 2, 3, 4, 5, 6, 7, 8, 8, d,\linebreak
11, 12, ..., 1d, 21, 22, ..., 2d, ..., 91, 92, ..., 9d, d1, d2, ..., dd,\linebreak
111, 112, ..., 11d, 121, 122, ..., 12d, ...\linebreak

La regla para conseguir el siguiente a una palabra $\alpha$ es la siguiente:
\begin{enumerate}
    \item si $\alpha = d^n$, con $n \geq 0$ entonces el siguiente de $\alpha$ es $1^{n+1}$
    \item si $\alpha$ no es de la forma $d^n$ entonces el siguiente de $\alpha$ se obtiene de la siguiente manera:
    \begin{enumerate}
        \item buscar de derecha a izquierda el primer símbolo no igual a $d$
        \item reemplazar dicho símbolo por su siguiente en la lista 1, 2, 3, 4, 5, 6, 7, 8, 9, $d$
        \item reemplazar por el símbolo 1 a todos los símbolos iguales a $d$ que ocurrían a la derecha del símbolo reemplazado
    \end{enumerate}
\end{enumerate}

Propiedades de la regla de asignación anterior:
\begin{itemize}
    \item (S) Toda palabra de $\widetilde{Num}^*$ aparece en la lista
    \item (I) Ninguna palabra de $\widetilde{Num}^*$ aparece más de una vez
\end{itemize}

Notar que la propiedad (S) nos dice que la función:
\hspace*{30ex} $*: \omega \rightarrow \widetilde{Num}^*$\linebreak
\hspace*{34ex}$n \mapsto (n+1)$-esimo elemento de la lista\linebreak
es sobreyectiva y la propiedad (I) nos garantiza que dicha funticón es inyectiva, por lo cual entre las dos garantizan que esta representación establece una biyección entre $\widetilde{Num}^*$ y $\omega$.\linebreak

Llamaremos $\#$ a la inversa de $*$, por lo que $\#(\alpha)$ es la posición que ocupa $\alpha$ en la lista de palabras de $\widetilde{Num}^*$ contando desde el 0, es decir, $\alpha$ es la $(\#(\alpha)+1)$-esima palabra de la lista.\linebreak

Tal como en el caso de la notación decimal, el número $\#(\alpha)$ se expresa como una suma de potencias de 10 (pues $|\widetilde{Num}| = 10$), con los coeficientes dados en función de los símbolos de $\alpha$. Más concretamente si $\alpha = s_1s_2...s_k$ con $k \geq 1$ y $s_1s_2...s_k \in \widetilde{Num}^*$, entonces:
\begin{center}
    $\#(\alpha) = \#(s_1).10^{k-1} + \#(s_2).10^{k-2} + ... + \#(s_k).10^0$ 
\end{center}

Dado que el siguiente a un elemento  $\alpha$ de la lista es de la misma longitud que $\alpha$ o tiene longitud igual a $|\alpha| + 1$, podemos representar a la lista anterior de la siguinte manera:
\begin{center}
    $B_0;B_1;B_2;...$
\end{center}
Donde cada $B_n$ es la parte de la lista en la cual las palabras tienen longitud exáctamente $n$. \linebreak

Propiedades básicas:
\begin{enumerate}
    \item Si $B_n = \alpha_1, ..., \alpha_k$ entonces $\alpha_1 = 1^n$ y $\alpha_k = d^n$ 
    \item Si $d^n$ ocurre en $B_n$ lo hace en la última posición
    \item Si $B_n = \alpha_1, ..., \alpha_k$, entonces $B_{n+1} = 1\alpha_1, ..., 1\alpha_k, 2\alpha_1, ..., 2\alpha_k, ..., d\alpha_1, ..., d\alpha_k$
    \item $B_n$ es una lista sin repeticiones de todas las palabras de longitud $n$
\end{enumerate}

\textbf{Lemma:} Sea $\sigma \in \widetilde{Num}$ y supongamos $\alpha \in \widetilde{Num}^*$ no es de la forma $d^n$. Entonces el sguiente a $\sigma\alpha$ es $\sigma\beta$ donde $\beta$ es el siguiente a $\alpha$.\pagebreak

\begin{Large}\textbf{Caso general}\end{Large}\linebreak

Sea $\Sigma$ un alfabeto no vacio y supongamos $\leq$ es un orden total sobre $\Sigma$. Supongamos que $\Sigma = \{a_1, ..., a_n\}$, con $a_1 < a_2 < ... < a_n$. Podemos dar la siguiente lista de palabras de $\Sigma^*$:\linebreak
$\varepsilon, a_1, a_2, ..., a_n$\linebreak
$a_1a_1, a_1a_2, ..., a_1a_n, a_2a_1, a_2a_2, ..., a_2a_n, ..., a_na_1,a_na_2, ..., a_na_n$\linebreak
$a_1a_1a_1, a_1a_1a_2, ..., a_1a_1a_n, a_1a_2a_1, a_1a_2a_2, ..., a_1a_2a_n, ..., a_1a_na_1, a_1a_na_2, ..., a_1a_na_n$\linebreak

Definamos $s^{\leq} : \Sigma^* \rightarrow \Sigma^*$ de la siguiente forma:\linebreak
\hspace*{7ex}- $s^{\leq}((a_n)^m) = (a_1)^{m+1}$\linebreak
\hspace*{7ex}- $s^{\leq}(\alpha a_i(a_n)^m) = \alpha a_{i+1}a_1^{m}$ cada vez que $\alpha \in \Sigma^*, 1 \leq i < n$ y $m \geq 0$. \linebreak

Propiedades:
\begin{itemize}
    \item $\varepsilon \neq s^{\leq}(\alpha), \forall \alpha \in \Sigma^*$
    \item Si $\alpha \neq \varepsilon$, entonces $\alpha = s^{\leq}(\beta)$ para algún $\beta \in \Sigma^*$
    \item $s^{\leq}(\alpha a_i) = \alpha a_{i+1}, i < n$
    \item $s^{\leq}(\alpha a_n) = s^{\leq}(\alpha)a_1$
\end{itemize}

Definamos $*^{\leq} : \omega \rightarrow \Sigma^*$ de la siguiente manera:\linebreak
\hspace*{7ex} -$*^{\leq}(0) = \varepsilon$\linebreak
\hspace*{7ex} -$*^{\leq}(i+1) = s^{\leq}(*^{\leq}(i))$\linebreak

$*^{\leq}(i)$ nos devuelve el $(i+1)$-esimo elemento de la lista, o lo que es lo mismo, el i-esimo elemento de la lista contando desde el 0.\linebreak

\textbf{Lemma:} Sea $\Sigma$ un alfabeto no vacio y supongamos $\leq$ es un orden total sobre $\Sigma$. Supongamos que $\Sigma = \{a_1, ..., a_n\}$, con $a_1 < a_2 < ... < a_n$. Entonces para cada $\alpha \in \Sigma^* - \{\varepsilon\}$ hay únicos $k \in \omega$ y $i_0, i_1, ..., i_k \in \{1, ..., n\}$ tales que:
\begin{center}
    $\alpha = a_{i_k}...a_{i_0}$
\end{center}
Notar que el $k$ del lemma anterior es $|\alpha| - 1$ y los números $i_k, ..., i_0$ van dando el número de orden de cada símbolo de $\alpha$ yendo de izquieda a derecha.\linebreak

Ahora definimos $\#^{\leq}$ (inversa de $*^{\leq}$) de la siguiente forma:\linebreak
\hspace*{7ex} - $\#^{\leq}: \Sigma^* \rightarrow \omega$\linebreak
\hspace*{16ex} $\varepsilon \mapsto 0$\linebreak
\hspace*{9ex} $a_{i_k}...a_{i_o} \mapsto i_kn^k + ... + i_0n^0$\linebreak

\textbf{Lemma:} Sea $n \geq 1$ fijo. Entonces cada $x \geq 1$ se escribe en forma única de la siguiente manera:
\begin{center}
    $x = i_kn^k + i_{k-1}n^{k-1} + ... + i_0n^0$
\end{center}
con $k \geq 0 $ y $1 \leq i_k, i_{k-1}, ..., i_0 \leq n$.\pagebreak

\begin{Large}\textbf{Extensión del orden total de $\Sigma$ a $\Sigma^*$}\end{Large}\linebreak

Podemos extender el orden de $\Sigma$ a $\Sigma^*$ de la siguiente manera:\linebreak
\hspace*{7ex} - $\alpha \leq \beta$ sii $\#^{\leq}(\alpha) \leq \#^{\leq}(\beta)$\linebreak
Es decir, $\alpha \leq \beta$ sii $\alpha$ ocurre antes que $\beta$ en la lista.\linebreak

\textbf{Lemma:} Si $S \subseteq \Sigma^*$ es no vacio, entonces $\exists \alpha \in S$ tal que $\alpha \leq \beta$ para cada $\beta \in S$.

\end{flushleft}
\end{document}
