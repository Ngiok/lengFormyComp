\documentclass{article}
\usepackage[utf8]{inputenc}

\title{Lenguajes Formales y Computabilidad\linebreak Definiciones, lemas y teoremas}
\author{Valentino Beorda}
\date{April 2022}

\begin{document}

\maketitle

\section{Funciones $\Sigma$-mixtas}

\begin{flushleft}

Una función es un conjunto $f$ de paers ordenados con la siguiente propiedad:
\begin{itemize}
\item Si $(x, y) \in f$ y $(x, z) \in f$, entonces  $y=z$.
\end{itemize}

$D_f$: dominio de $f$ = $\{x : (x, y) \in f$ para algún $y\}$\linebreak
$I_f$: imagen de $f$ = $\{y : (x, y) \in f$ para algún $x\}$\linebreak

\textbf{Igualdad de funciones:} Sean $f$ y $g$ funciones. Entonces $f = g$ sii $D_f = D_g$ y para cada $x \in D_f$ se tiene que $f(x) = g(x)$.\linebreak

\textbf{Definición:}
Sea $\Sigma$ un alfabeto finito. Dada una función $f$, diremos que $f$ es $\Sigma$-mixta si cumple las siguientes propiedades:
\begin{enumerate}
\item Existen $n$, $m$ $\geq$ 0, tales que $D_f$ $\subseteq \omega^n \times \Sigma^{*m}$
\item $I_f \subseteq \omega$ ó $I_f \subseteq \Sigma^*$.
\end{enumerate}

\textbf{Lemma:} Supongamos que $\Sigma \subseteq \Gamma$ son alfabetos finitos. Entonces si $f$ es una función $\Sigma$-mixta, $f$ es $Gamma$-mixta\linebreak

\textbf {Funciones Útiles:}\linebreak

Sucesor: $Suc$: $\omega \rightarrow \omega$\linebreak
\hspace*{14ex} $n \mapsto n+1$\linebreak

Predecesor: $Pred: $ \textbf{N} $\rightarrow \omega$\linebreak
\hspace*{20ex} $n \mapsto n-1$\linebreak

Derecha sub a: $d_a: \Sigma^* \rightarrow \Sigma^*$\linebreak
\hspace*{21ex} $\alpha \mapsto \alpha a$\linebreak

Proyecciones: \linebreak
para un $i$ tal que $1 \geq i \geq n$,\linebreak
$p_i^{n, m}: \omega^n \times \sigma^{*m} \rightarrow \omega$\linebreak
\hspace*{8ex} $(\overrightarrow{x}, \overrightarrow{\alpha}) \mapsto x_i$\linebreak
Para un $i$ tal que $n \geq i \geq n+m$, \linebreak
$p_i^{n, m}: \omega^n \times \sigma^{*m} \rightarrow \omega$\linebreak
\hspace*{8ex} $(\overrightarrow{x}, \overrightarrow{\alpha}) \mapsto \alpha_{i-n}$\linebreak

Constantes: $C_k^{n, m}: \omega^n \times \Sigma^{*m} \rightarrow \omega$ \hspace*{13ex}$C_\alpha^{n, m}: \omega^n \times \Sigma^{*m} \rightarrow
\Sigma^*$\linebreak
\hspace*{22ex} $(\overrightarrow{x}, \overrightarrow{\alpha}) \mapsto k $ \hspace*{22ex} $(\overrightarrow{x}, \overrightarrow{\alpha}) \mapsto \alpha $\linebreak\linebreak

\textbf{Tipo de una función mixta:} Dada $f$ una función $\Sigma$-mixta, si $n, m \in \omega$ son tales que $D_f \subseteq \omega^n \times \Sigma^{*m}$ y además $I_f \subseteq \omega$, diremos $f$ es una función de tipo $(n, m, \#)$; si $I_f \subseteq \Sigma^*$, entonces $f$ es de tipo $(n, m, *)$. \linebreak

\textbf{Predicados $\Sigma$-mixtos:} Un predicado $\Sigma$ mixto es una función $f$ la cual es $\Sigma$-mixta y además cumple $I_f \subseteq \{0,1\}$.\linebreak

\textbf{Función identidad:} Dado un conjunto A, a la función:
\begin{center}
$A \rightarrow A$\linebreak
$a \mapsto a$
\end{center}
La denotaremos con $Id_A$ y la llamaremos la función identidad sobre A. Notar que $Id_a = \{(a, a): a \in A\}$ \linebreak.

\textbf{Composición de funciones:} Dadas funciones $f$ y $g$ definamos la función $f \circ g$ de la siguiente manera:
\begin{itemize}
\item $D_{f \circ g} = \{ e \in D_g : g(e) \in D_f\}$
\item $f \circ g(e) = f(g(e))$
\end{itemize}

Notar que $f \circ g = \{(u, v): \exists$ z tal que $ (u, z) \in g$ y $(z, v) \in f\}.$\linebreak

\textbf{Lemma:} $f \circ g \neq \emptyset$ si y solo si $I_g \cap D_f \neq \emptyset.$\linebreak

\textbf{Lemma:} Si $I_g \in D_f \Rightarrow D_{f \circ g} = D_g$ \linebreak

\textbf{Funciones de la forma $[f_1,...,f_n]$}: Dadas funciones $f_1, ..., f_n$ con $ n \geq 2$, definimos la función $[f_1, ..., f_n]$ de la siguiente forma:
\begin{itemize}
\item $D_{[f1, ..., fn]} = D_{f_1} \cap ... \cap D_{f_n}$
\item $[f_1, ..., f_n](e) = (f_1(e), ..., f_n(e))$
\end{itemize}
Notar que $I_{[f_1, ..., f_n]} \in I_{f_1} \times ... \times I_{f_n}.$\linebreak

\textbf{Función inyectiva:} Una función $f$ es inyectiva cuando \textbf{no} se da que $f(a) = f(b)$ $\forall$ $a, b \in D_f$ con $a \neq b$.\linebreak
\textbf{Función suryectiva:} Una función $f: A \rightarrow B$ es suryectiva cuando $I_f = B$.\linebreak
\textbf{Función biyectiva:} Una función $f$ es biyectiva, cuando $f$ es inyectiva y suryectiva.\linebreak
\textbf{Función inversa:} Si $f$ es biyectiva, entonces se puede definir $f^{-1}: B \rightarrow A$, de la siguiente manera:
\begin{center}
    $f^{-1}(b) = $ único $a \in A$ tal que $f(a) = b$
\end{center}
La función $f^{-1}$ será llamada inversa de $f$. Notar que $f \circ f^{-1} = Id_B$ y $f^{-1} \circ f = Id_A$.\linebreak

\textbf{Lemma:} Supongamos $f: A \rightarrow B$ y $g: B \rightarrow A$ son tales que $f \circ g = Id_B$ y $g \circ f = Id_A$. Entonces $f$ y $g$ son biyectivas, $f^{-1} = g$ y $g^{-1} = f$.\linebreak

\textbf{Conjuntos $\Sigma$-mixtos:} Un conjunto $S$ es llamado $\Sigma$-mixto si existen $n, m \in \omega$ tales que $S \subseteq \omega^n \times \Sigma^{*m}$.\linebreak

\textbf{Tipo de un conjunto $\Sigma$-mixto:} Dado un conjunto $\Sigma$-mixto $S$, si $n, m \in \omega$ son tales que $S \subseteq \omega^n \times \Sigma^{*m}$, entonces diremos que $S$ es de tipo $(n, m)$.\linebreak

\textbf{Notación Lambda:} \linebreak

Para que una expresión $E$ pueda ser utilizada en la notación lambda, deberá cumplir alguna de estas 2 propiedades:
\begin{enumerate}
    \item Los valores que asuma $E$ deberán ser elementos de $\omega$
    \item Los valores que asuma $E$ deberán ser elementos de $\Sigma^*$.
\end{enumerate}

\textbf{Definión de $\lambda_{x_1, ..., x_n, \alpha_1, ..., \alpha_m}[E]$:} Sea $x_1, ..., x_n$ una lista de variables todas distintas tal que las variables numéricas que ocurren en $E$ estén todas contenidas en $x_1, ..., x_n$ y que las variables alfabéticas que ocurren en $E$, estén todas contenidas en $\alpha_1, ..., \alpha_m$. Entonces $\lambda_{x_1, ..., x_n, \alpha_1, ..., \alpha_m}[E]$ denotará la función dada por:
\begin{enumerate}
    \item El dominio de $\lambda_{x_1, ..., x_n, \alpha_1, ..., \alpha_m}[E]$ es el conjunto de las $(n+m)$-uplas $(k_1, ..., k_n, \beta_1, ..., \beta_m)$ tales que $E$ está definida cuando le asignamos a cada valor $x_i$ el valor $k_i$ y a cada $\alpha_i$ el valor $\beta_i$.
    \item $\lambda_{x_1, ..., x_n, \alpha_1, ..., \alpha_m}[E](k_1, ..., k_n, \beta_1, ..., \beta_m)$ = valor que asume o representa $E$ cuando le asignamos a cada $x_i$ el valor $k_i$ y a cada $\alpha_i$ el valor $\beta_i$.
\end{enumerate}

Notar que para $S \subseteq \omega^n \times \Sigma^{*m}$ se tiene que $\chi_S^{\omega^n \times \Sigma^{*m}} = \lambda_{x_1, ..., x_n, \alpha_1, ..., \alpha_m}[(\overrightarrow{x}, \overrightarrow{\alpha}) \in S]$.

\end{flushleft}
\end{document}
